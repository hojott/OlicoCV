%-----------------------------------------------------------------------------------------------------------------------------------------------%
% The MIT License (MIT)
%
% Copyright (c) 2023
%
% Permission is hereby granted, free of charge, to any person obtaining a copy
% of this software and associated documentation files (the "Software"), to deal
% in the Software without restriction, including without limitation the rights
% to use, copy, modify, merge, publish, distribute, sublicense, and/or sell
% copies of the Software, and to permit persons to whom the Software is
% furnished to do so, subject to the following conditions:
% 
% THE SOFTWARE IS PROVIDED "AS IS", WITHOUT WARRANTY OF ANY KIND, EXPRESS OR
% IMPLIED, INCLUDING BUT NOT LIMITED TO THE WARRANTIES OF MERCHANTABILITY,
% FITNESS FOR A PARTICULAR PURPOSE AND NONINFRINGEMENT. IN NO EVENT SHALL THE
% AUTHORS OR COPYRIGHT HOLDERS BE LIABLE FOR ANY CLAIM, DAMAGES OR OTHER
% LIABILITY, WHETHER IN AN ACTION OF CONTRACT, TORT OR OTHERWISE, ARISING FROM,
% OUT OF OR IN CONNECTION WITH THE SOFTWARE OR THE USE OR OTHER DEALINGS IN
% THE SOFTWARE.
% 
%
%-----------------------------------------------------------------------------------------------------------------------------------------------%




%============================================================================%
%
% DOCUMENT DEFINITION
%
%============================================================================%



%we use article class because we want to fully customize the page and don't use a cv template
\documentclass[10pt]{article} 

\renewcommand{\section}[1]{} % I dont want sections
\renewcommand{\subsection}[1]{} % I dont want subsections


\section{Preamble}
\subsection{packages}

\usepackage{eurosym} % for \euro

\usepackage{fancyhdr} % nom sur le dessus
\usepackage{lastpage} % numéro de page à la fin (référence)

%----------------------------------------------------------------------------------------
% ENCODING
%----------------------------------------------------------------------------------------

% we use utf8 since we want to build from any machine
\usepackage[utf8]{inputenc}  

%----------------------------------------------------------------------------------------
% LOGIC
%----------------------------------------------------------------------------------------

% provides \isempty test
\usepackage{xstring, xifthen}

%----------------------------------------------------------------------------------------
% FONT BASICS
%----------------------------------------------------------------------------------------

% some tex-live fonts - choose your own

%\usepackage[defaultsans]{droidsans}
%\usepackage[default]{comfortaa}
%\usepackage{cmbright}
%\usepackage[default]{raleway}
%\usepackage{fetamont}
\usepackage[default]{gillius}
%\usepackage[light,math]{iwona}
%\usepackage[thin]{roboto} 

% set font default
\renewcommand*\familydefault{\sfdefault}  
\usepackage[T1]{fontenc}

% more font size definitions
\usepackage{moresize}

%----------------------------------------------------------------------------------------
% FONT AWESOME ICONS
%---------------------------------------------------------------------------------------- 

% include the fontawesome icon set
\usepackage{fontawesome5}

% use to vertically center content
% credits to: http://tex.stackexchange.com/questions/7219/how-to-vertically-center-two-images-next-to-each-other
\newcommand{\vcenteredinclude}[1]{\begingroup
\setbox0=\hbox{\includegraphics{#1}}%
\parbox{\wd0}{\box0}\endgroup}

% use to vertically center content
% credits to: http://tex.stackexchange.com/questions/7219/how-to-vertically-center-two-images-next-to-each-other
\newcommand*{\vcenteredhbox}[1]{\begingroup
\setbox0=\hbox{#1}\parbox{\wd0}{\box0}\endgroup}

% icon shortcut
\newcommand{\icon}[3] {        
 \makebox(#2, #2){\textcolor{maincol}{\faIcon{#1}}}
} 

% icon with text shortcut
\newcommand{\icontext}[4]{       
 \vcenteredhbox{\icon{#1}{#2}{#3}}  \hspace{2pt}  \parbox{0.9\mpwidth}{\textcolor{#4}{#3}}
}

% icon with website url
\newcommand{\iconhref}[5]{       
    \vcenteredhbox{\icon{#1}{#2}{#5}}  \hspace{2pt} \href{#4}{\textcolor{#5}{#3}}
}

% icon with email link
\newcommand{\iconemail}[5]{       
    \vcenteredhbox{\icon{#1}{#2}{#5}}  \hspace{2pt} \href{mailto:#4}{\textcolor{#5}{#3}}
}

%----------------------------------------------------------------------------------------
% PAGE LAYOUT  DEFINITIONS
%----------------------------------------------------------------------------------------
\subsection{page layout}
% page outer frames (debug-only)
% \usepackage{showframe}  

% we use paracol to display breakable two columns
\usepackage{paracol}

% define page styles using geometry
\usepackage[a4paper]{geometry}

% remove all possible margins
%\geometry{top=1cm, bottom=1cm, left=1cm, right=1cm}


% remove all possible margins
\geometry{top=1cm, bottom=1.50cm, left=1cm, right=1cm, footskip=1cm}

\usepackage{fancyhdr}
\renewcommand{\headrulewidth}{0pt} 

% space between header and content
% \setlength{\headheight}{0pt}

% indentation is zero
\setlength{\parindent}{0mm}

%----------------------------------------------------------------------------------------
% TABLE /ARRAY DEFINITIONS
%---------------------------------------------------------------------------------------- 

% extended aligning of tabular cells
\usepackage{array}

% custom column right-align with fixed width
% use like p{size} but via x{size}
\newcolumntype{x}[1]{%
>{\raggedleft\hspace{0pt}}p{#1}}%


%----------------------------------------------------------------------------------------
% GRAPHICS DEFINITIONS
%---------------------------------------------------------------------------------------- 

%for header image
\usepackage{graphicx}

% use this for floating figures
% \usepackage{wrapfig}
% \usepackage{float}
% \floatstyle{boxed} 
% \restylefloat{figure}

%for drawing graphics  
\usepackage{tikz}    
\usetikzlibrary{shapes, backgrounds,mindmap, trees}

%----------------------------------------------------------------------------------------
% Color DEFINITIONS
%---------------------------------------------------------------------------------------- 
\usepackage{transparent}
\usepackage{color}
\usepackage{xhfill} % for the line 
% primary color
\definecolor{maincol}{RGB}{227,5, 19} 

% dark color
\definecolor{darkcol}{RGB}{ 70, 70, 70 }

% light color
\definecolor{lightcol}{RGB}{220,220,220}


% Package for links, must be the last package used
\usepackage[hidelinks]{hyperref}

% returns minipage width minus two times \fboxsep
% to keep padding included in width calculations
% can also be used for other boxes / environments
\newcommand{\mpwidth}{\linewidth-\fboxsep-\fboxsep}
 


%============================================================================%
%
% CV COMMANDS
%
%============================================================================%

%----------------------------------------------------------------------------------------
%  CV LIST
%----------------------------------------------------------------------------------------

% renders a standard latex list but abstracts away the environment definition (begin/end)
\newcommand{\cvlist}[1] {
 \begin{itemize}{#1}\end{itemize}
}

%----------------------------------------------------------------------------------------
%  CV TEXT
%----------------------------------------------------------------------------------------

\subsection{CV text}

% base class to wrap any text based stuff here. Renders like a paragraph.
% Allows complex commands to be passed, too.
% param 1: *any
\newcommand{\cvtext}[1] {
 \begin{tabular*}{1\mpwidth}{p{0.98\mpwidth}}
  \parbox{1\mpwidth}{#1}
 \end{tabular*}
}

%----------------------------------------------------------------------------------------
% CV SECTION
%----------------------------------------------------------------------------------------
\subsection{CV section}
% Renders a a CV section headline with a nice underline in main color.
% param 1: section title
\newcommand{\cvsection}[1] {
 \vspace{14pt}
 \cvtext{
  \textbf{\LARGE{\textcolor{darkcol}{\uppercase{#1}}}}\\[-4pt]
  \textcolor{maincol}{ \rule{0.15\textwidth}{2pt}}  \\
    }
    % \xrfill[0ex]{2pt}[lightcol] %to add at 250
}

%----------------------------------------------------------------------------------------
% META SKILL
%----------------------------------------------------------------------------------------

\subsection{CV skill}
% Renders a progress-bar to indicate a certain skill in percent.
% param 1: name of the skill / tech / etc.
% param 2: level (for example in years)
% param 3: percent, values range from 0 to 1
\newcommand{\cvskill}[3] {
 \begin{tabular*}{1\mpwidth}{p{0.53\mpwidth}  r}
   \textcolor{black}{\textbf{#1}} & \textcolor{maincol}{#2}\\
 \end{tabular*}%
 
 \hspace{4pt}
 \begin{tikzpicture}[scale=1,rounded corners=2pt,very thin]
  \fill [lightcol] (0,0) rectangle (1\mpwidth, 0.15);
  \fill [maincol] (0,0) rectangle (#3\mpwidth, 0.15);
   \end{tikzpicture}%
}


%----------------------------------------------------------------------------------------
%  CV EVENT
%----------------------------------------------------------------------------------------
\subsection{CV event}
% Renders a table and a paragraph (cvtext) wrapped in a parbox (to ensure minimum content
% is glued together when a pagebreak appears).
% Additional Information can be passed in text or list form (or other environments).
% the work you did
% param 1: time-frame i.e. Sep 14 - Jan 15 etc.
% param 2:  event name (job position etc.)
% param 3: Customer, Employer, Industry
% param 4: Short description
% param 5: work done (optional)
% param 6: technologies include (optional)
% param 7: achievements (optional)
\newcommand{\cvevent}[7] {
 
 % we wrap this part in a parbox, so title and description are not separated on a pagebreak
 % if you need more control on page breaks, remove the parbox
 \parbox{\mpwidth}{
  \begin{tabular*}{1\mpwidth}{p{0.78\mpwidth}  r}
    \textcolor{black}{\textbf{#2}} \xrfill[.5ex]{2pt}[lightcol] & \colorbox{maincol}{\makebox[0.20\mpwidth]{\textcolor{white}{#1}}} \\
   \textcolor{maincol}{\textbf{#3}} & \\
  \end{tabular*}\\[5pt]
  \ifthenelse{\isempty{#4}}{}{
   \cvtext{#4}\\
  }
 }

 \ifthenelse{\isempty{#5}}{}{
  \vspace{9pt}
  \cvtext{#5}
 }

 \ifthenelse{\isempty{#6}}{}{
  \vspace{9pt}
  \cvtext{\textbf{Technologies include:}}\\
  \cvtext{#6}
 }

 \ifthenelse{\isempty{#7}}{}{
  \vspace{9pt}
  \cvtext{\textbf{Achievements include:}}\\
  \cvtext{#7}
 }
 \vspace{10pt}
}

%----------------------------------------------------------------------------------------
%  CV META EVENT
%----------------------------------------------------------------------------------------

\subsection{CV metaevent}
% Renders a CV event on the sidebar
% param 1: title
% param 2: subtitle (optional)
% param 3: customer, employer, etc,. (optional)
% param 4: info text (optional)
\newcommand{\cvmetaevent}[4] {
 \textcolor{maincol} {\cvtext{\textbf{\begin{flushleft}#1\end{flushleft}}}}

 \ifthenelse{\isempty{#2}}{}{
 \textcolor{darkcol} {\cvtext{\textbf{#2}} }
 }

 \ifthenelse{\isempty{#3}}{}{
  \cvtext{{ \textcolor{darkcol} {#3} }}\\
 }

 \cvtext{#4}\\[14pt]
}

%---------------------------------------------------------------------------------------
% QR CODE
%----------------------------------------------------------------------------------------

\subsection{Code QR}
% Renders a qrcode image (centered, relative to the parentwidth)
% param 1: percent width, from 0 to 1
\newcommand{\cvqrcode}[1] {
 \begin{center}
  \includegraphics[width={#1}\mpwidth]{qrcode.jpg}
 \end{center}
}

\section{To do}
%%%%%%%%%%%%%%%%%%%%%%%%%%%% TODO %%%%%%%%%%%%%%%%%%%%%%%%%%%
%


%============================================================================%
%
%
%
% DOCUMENT CONTENT
%
%
%
%============================================================================%
\begin{document}
\section{Document}
\subsection{Settings}
%%% Footer
\pagestyle{fancy}
\fancyfoot[L]{\small \textcolor{black!20!lightcol}{Compilation : \today}}
\fancyfoot[C]{\small  \textcolor{black!20!lightcol}{\thepage \hspace{1pt} of  \pageref{LastPage}}}


\columnratio{0.33, 0.33, 0.33} %0.31
\setlength{\columnsep}{2.2em}
\setlength{\columnseprule}{4pt}
\colseprulecolor{lightcol}

%---------------------------------------------------------------------------------------
% TITLE  HEADER
%----------------------------------------------------------------------------------------
 \section{Header}
%  \begin {center}
\fcolorbox{white}{lightcol!50}{\begin{minipage}[c][3cm][c]{1\mpwidth}
 \begin {center}
  \HUGE{ \textbf{ \textcolor{darkcol}{ \uppercase{ SAKARI MARTTINEN } } } } \\[-24pt]
  \textcolor{darkcol}{ \rule{0.1\textwidth}{1.25pt} } \\[4pt]
  \large{ \textcolor{darkcol} {Curriculum vitae} }
 \end {center}
\end{minipage}} \\[14pt]
%  \end {center}
\vspace{12pt}



\section{columns}
\begin{paracol}{3}
\begin{column}

\subsection{Contact information}
\cvsection{Contact \\[2pt] information}
 
\icontext{map-marker-alt}{12}{Helsinki, Finland}{black}\\[6pt]
\iconemail{envelope}{12}{sakari.marttinen@helsinki.fi}{sakari.marttinen@helsinki.fi}{black}\\[6pt]
\iconhref{github}{12}{hojott}{https://github.com/hojott}{black}\\[6pt]
\iconhref{linkedin}{12}{Sakari Marttinen}{https://www.linkedin.com/in/kussi}{black}\\[6pt]




\end{column}
\begin{column}


%---------------------------------------------------------------------------------------
% META SKILLS
%----------------------------------------------------------------------------------------
%\cvqrcode{0.5}

\subsection{Skills}
\cvsection{Skills}

\cvskill{Version Control} {4 years} {1} \\[-2pt]

\cvskill{Programming} {4 years actively} {0.8} \\[-2pt]

\cvskill{SysAdmin} {3 years} {0.7} \\[-2pt]


\end{column}
\begin{column}
\subsection{Languages}
\cvsection{Languages}

\cvskill{Finnish} {Mother tongue} {1} \\[-2pt]

\cvskill{English} {Advanced} {0.9} \\[-2pt]
\end{column}
\end{paracol} 

\vspace{14pt}


\section{Education}
\cvsection{Education}


\cvevent
{2023 -> }
{Bachelor of Computer science}
{University of Helsinki}
{Also doing my minor on astronomy :>}{}{}{}


%---------------------------------------------------------------------------------------
% WORK EXPERIENCE
%----------------------------------------------------------------------------------------

\section{Work experiences}
\cvsection{WORK EXPERIENCEs}


\cvevent
 {Apr 2024 - Nov 2024}
 {Head System Administrator}
 {Päivölä institute}
 {The work of an allround sysadmin includes helping students and teachers with their digital problems, like working with projectors, printers, SIM-cards and Linux and Windows computers. On a hardware level it includes updating modems and fixing switches around the campus, in addition to installing new ones. On a server level it includes updating webapps and laptops using Ansible, and fixing problems with Docker and HetznerCloud platform. It also included bits of web development with React, to develop our own webapps.}{
 Also as the head of a small team of students, I had to teach these technologies, give out tasks and make sure they are done.
 }{
 React, Docker, Ansible, Linux, Windows
 }{}

\cvevent
 {Jul 2023}
 {Computer science teacher}
 {Päivölä institute}
 {I was the senior of two teachers at a computer science camp for highschoolers. It required lots of responsibility, having to create plans and make sure everything is going well. I scrathed the surfaces of many different topics and taught Python}{}{
 Python
 }{}

 \cvevent
 {May 2022 - Aug 2023}
 {Junior System Administrator}
 {Päivölä institute}
 {As a junior I did very similar stuff to as when I worked as the head. The difference was I was taught at the same time, and I did it fully from my own interest}{}{
 React, Docker, Ansible, Linux, Windows
 }{}

 \cvevent
 {Aug 2021 - Jun 2023}
 {Software developer}
 {Päivölä Student Innovation Lab}
 {In PSIL I worked in a team of other students. The most important skills I learned was version control and working with others. We used Scrum for 2 years, and were taught to use version control in a civilized way.}{
 I worked in 3 different projects: The first was web development, where our task was to modify and update an existing project created with Rust. The second project used Python to control and read sensors, like an infrared-camera. In the last project I was the lead, where we created a file-encryption webapp with Rust, WebAssembly and React.
 }{
 React, Flexbox, Python, Numpy, Rust, WebAssembly
 }{}


%---------------------------------------------------------------------------------------
% Projexts
%----------------------------------------------------------------------------------------

\section{Projects}
\cvsection{Projects}


\cvevent
 {Jan 2023 ->}
 {My Website}
 {https://github.com/hojott/www/}
 {My website: www.kussi.fi! Nothing too fancy. It does include a portfolio page :) There’s also a pile of branches in Github that I never finished.}{}{
 Pure Typescript
 }{}

\cvevent
 {Dec 2022 ->}
 {My Servers}
 {not yet on github}
 {I host many websites and a few Minecraft servers on my servers. I currently have an Oracle server and a Hetzner server, both running with docker compose.I host many websites and a few Minecraft servers on my servers. I currently have an Oracle server and a Hetzner server, both running with docker compose.}{}{
 Docker, Docker compose
 }{}

\cvevent
 {Oct 2024}
 {Kjyr-TJ bot}
 {https://github.com/hojott/kjyrtj\_bot/}
 {KJYR, the fresher cruise, is Kumpula’s most anticipated event (right after WappuHopLop). This is just a simple Telegram bot that responds to commands with the number of days left until the cruise. I included it because it’s one of the few programs I actually wrote to completion. Eventually, someone with their own superior kjyrtj-bot made n years ago showed up, and mine was depracated. Still, mine was in use for like 1 or 2 days.}{}{
 Telegram, Chatbots, APIs
 }{}


\cvevent
 {Nov 2024}
 {Victoria 2 Analyzer}
 {https://github.com/hojott/victoria2-analyzer/}
 {Victoria 2 is a grand strategy game from around 2010 by Paradox Interactive, set in the 1800–1900s. If you start a world war in it, you can rack up massive casualties in battles, but normally you never get to see how many died in total. This program parses the savegame file and counts the number of deaths. There exists a Java implementation of something similar, but I couldn’t get it to work on Linux :D The coolest part of this is probably the parser, because the actual save files have no actual format and are an absolute nightmare to go through.}{}{
 Game Development, Parsing
 }{}

\cvevent
 {Sep 2023}
 {Glögi compiler}
 {https://github.com/hojott/glogicompiler/}
 {I tried to make my own compiler at the start of my fresher year. My biggest mistake was probably trying to write it in zsh (yeah, not even bash :D). Still, I got pretty far with it, and at least I ended up doing a lot of shell scripting. I left it at the point where I was supposed to implement variable evaluation, which sounds easier than it actually is.}{}{
 Bash, Zsh, Shell Scripting
 }{}

%---------------------------------------------------------------------------------------
% Volunteering
%----------------------------------------------------------------------------------------
\section{Volunteering}
\cvsection{Volunteering}

\cvevent
 {2025}
 {Board member}
 {TKO-äly ry}
 {The board of TKO-äly manages it's resources and bureocracy, represents and is responsible for the organization and overall does a lot of work behind the scenes}{}{}{}

\cvevent
 {Fall 2025}
 {Hostess}
 {Savolainen Osakunta}
 {I choose the menu for table parties, and lead the team cooking it.}{}{}{}

\cvevent
 {2024, 2025}
 {Tutor}
 {TKO-äly ry}
 {Tutoring freshers and orienting them into universitylife.}{}{}{}

\cvevent
 {2025}
 {Casual Officer}
 {TKO-äly ry}
 {I hold Casual Nights, where people for example play board games}{}{}{}

\cvevent
 {2024}
 {Clubroom Officer}
 {TKO-äly ry}
 {Keeping our main clubroom nice and clean. Also helping people around in there.}{}{}{}

%---------------------------------------------------------------------------------------
% References
%----------------------------------------------------------------------------------------
\section{References}
\cvsection{References}

\cvevent
 {:>}
 {Joel Jäkkö}
 {Project Manager, Päivölä Student Innovation Lab}
 {\iconemail{envelope}{12}{joel.jakko@psil.fi}{joel.jakko@psil.fi}{black}\\[6pt]
\icontext{phone}{12}{+358 440 484145}{black}\\[6pt]}{}{}{}

\cvevent
 {:>}
 {Ville Nupponen}
 {Head Administrator, Päivölä Institute}
 {\iconemail{envelope}{12}{ville.nupponen@paivola.fi}{ville.nupponen@paivola.fi}{black}\\[6pt]
\icontext{phone}{12}{+358 441 3001}{black}\\[6pt]}{}{}{}


\end{document}
